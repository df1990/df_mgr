% Chapter Template

\chapter{Wnioski} % Main chapter title

\label{Chapter8} % Change X to a consecutive number; for referencing this chapter elsewhere, use \ref{ChapterX}

\lhead{Chapter 8. \emph{Wnioski}} % Change X to a consecutive number; this is for the header on each page - perhaps a shortened title

Celem niniejszej pracy było zapoznanie się z zasadą działania i stabilizacją lotu czterowirnikowych śmigłowców. Jako że autor pracy w chwili rozpoczęcia realizacji projektu nie posiadał żadnego doświadczenia w tworzeniu zdalnie sterowanych robotów, nacisk położono na prostotę projektowanych rozwiązań. Docelowo starano się osiągnąć prostą i~uniwersalną konstrukcję, stanowiącą punkt wyjściowy dla przyszłych wersji kwadrokoptera.

W ramach pracy wykonano następujące zadania:
\begin{itemize}
	\item Opracowano projekt kwadrokoptera:
	\begin{itemize}
		\item Opracowano modułową koncepcję urządzenia z podziałem na kontroler lotu i sterowniki silników
		\item Określono liczbę procesorów oraz określono zadania, za które są odpowiedzialne
		\item Stworzono schematy elektryczne oraz projekty płytek drukowanych dla układów kontrolera lotu i sterowników silników
		\item Dokonano montażu prototypu
		\item Wykorzystując dostępne biblioteki programowe, napisano algorytm stabilizacji i kontroli lotu kwadrokoptera
		\item Opracowano binarny protokół używany do komunikacji z kwadrokopterem
	\end{itemize}
	\item Napisano aplikacje uzytkownika:
	\begin{itemize}
		\item Aplikacja przeznaczona na platformę Android, służąca do kontroli kwadrokoptera
		\item Aplikacja przeznaczona na platformę PC, służąca do testów oprogramowania kwadrokoptera oraz strojenia parametrów algorytmu kontroli lotu
	\end{itemize}
	\item Opracowano stanowisko testowe, przeznaczone do strojenia algorytmu kwadrokoptera	
	\item Dokonano testów wszystkich zaprojektowanych modułów elektronicznych oraz aplikacji
	\item Dokonano strojenia algorytmu kwadrokoptera
\end{itemize}

Efektem wykonania zadań wymienionych powyżej jest opracowanie kwadrokoptera będącego uniwersalnym modułowym urządzeniem, które w prosty sposób może być modyfikowane i rozszerzane o nowe moduły elektroniczne (np. nowe moduły transmisji radiowej, dodatkowe moduły czujników). 
Zaprojektowane urządzenie spełnia wszystkie założenia projektowe i wymagania techniczne, co dowodzi słuszności zastosowanych rozwiązań.
W ramach projektu stworzono dwie wersje urządzenia. Błędy wykryte w pierwszej rewizji kwadrokoptera zostały poprawione w rewizji drugiej, która po przejściu testów układów elektronicznych oraz oprogramowania stała się finalną rewizją prezentowaną w pracy. 

