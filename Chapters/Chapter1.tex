% Chapter Template

\chapter{Wstęp} % Main chapter title
\setcounter{page}{7}
\label{Chapter1} % Change X to a consecutive number; for referencing this chapter elsewhere, use \ref{ChapterX}

\lhead{Chapter 1. \emph{Wstęp}} % Change X to a consecutive number; this is for the header on each page - perhaps a shortened title

Rozwój techniki cyfrowej w trakcie ostatniej dekady doprowadził do powstania mikrokontrolerów, będących bardzo silną gałęzią ryknu układów cyfrowych. Do ich niezaprzeczalnych zalet należą takie cechy jak:

\begin{itemize}
	\item Niski pobór mocy
	\item Obecność rozmaitych sprzętowych interfejsów komunikacyjnych, do których zaliczają się:
		\begin{itemize}
			\item UART
			\item I\textsuperscript{2}C
			\item SPI
		\end{itemize}
	\item Obecność sprzętowych modułów liczników, zdolnych do generowania wielu sygnałów PWM jednocześnie
	\item Obecność przetworników analogowo-cyfrowych oraz coraz częściej cyfrowo-analogowych
	\item Dostępność układów w obudowach o niewielkich wymiarach
	\item Niska cena jednostkowa 
\end{itemize}

Zalety te doprowadziły do konsekwentnego wzrostu popularności mikrokontrolerów w ciągu ostatnich lat, oraz do stosowania ich w rozmaitych aplikacjach, do których zaliczają się między innymi aplikacje mobilne takie jak autonomiczne i zdalnie sterowane roboty.
Robotami do konstrukcji których konstruktorzy bardzo chętnie stosują mikrokontrolery jest prężnie rozwijająca się ostatnio rodzina czterowirnikowych śmigłowców zwanych również kwadrokopterami. 

Cechy takie jak prostota oraz wytrzymałość konstrukcji mechanicznej, możliwość pionowego startu i lądowania a także umiejętność wykonywania rozmaitych manewrów powietrznych sprawiły, że grono użytkowników kwadrokopterów stale się poszerza od chwili ich pojawienia się na rynku.
Należą do niego już nie tylko piloci startujący w zawodach zdalnie sterowanych modeli latających, lecz coraz częściej firmy używające kwadrokopterów do realizacji takich zadań jak:
%  kwadrokoptery od swojego pojawienia się na rynku znajdują wśród stale rosnącego grona użytkowników coraz to nowe zastosowania, do których zaliczyć można między innymi:

\begin{itemize}
	\item Nagrywanie filmów kręconych z powietrza
	\item Patrolowanie niewielkich terenów
	\item Transport niewielkich ładunków
\end{itemize}
 
Widać zatem, że kwadrokoptery mają duży potencjał komercyjny, a co za tym idzie ich rozwijający się rynek będzie stale potrzebował nowych rozwiązań mechanicznych, elektronicznych oraz programistycznych. Stało się to, w połączeniu z faktem, iż kwadrokoptery same w sobie stanowią interesujący projekt z dziedziny elektroniki oraz programowania, główną motywacją do opracowania autorskiej konstrukcji czterowirnikowego śmigłowca.

 
